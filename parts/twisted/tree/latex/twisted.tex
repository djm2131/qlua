\documentclass[twoside,openright,letterpaper]{article}
\usepackage[boxed]{algorithm2e}
\usepackage{color}
\definecolor{darkblue}{cmyk}{1,1,0,0.7}
\usepackage[dvipdfm,colorlinks=true,linkcolor=darkblue]{hyperref}
\newcommand{\note}[1]{$[\![$NB: #1$]\!]$}
\setlength{\parindent}{0pt}
\setlength{\topmargin}{-40pt}
\setlength{\oddsidemargin}{-16pt}
\setlength{\evensidemargin}{-16pt}
\setlength{\textwidth}{522pt}
\setlength{\textheight}{670pt}
%\setcounter{secnumdepth}{3}
%\setcounter{tocdepth}{3}
\newcommand{\Cee}{\texttt{C}}

\title{Twisted Mass Wilson Fermions Implementation}
\author{Andrew Pochinsky\\Sergey Syritsyn}

\begin{document}
\maketitle
\thispagestyle{empty}\hbox{}
\vfill
\copyright{} 2015 Massachusetts Institute of Technology

Permission is hereby granted, free of charge, to any person obtaining
a copy of this software and associated documentation files (the
"Software"), to deal in the Software without restriction, including
without limitation the rights to use, copy, modify, merge, publish,
distribute, sublicense, and/or sell copies of the Software, and to
permit persons to whom the Software is furnished to do so, subject to
the following conditions:

The above copyright notice and this permission notice shall be
included in all copies or substantial portions of the Software.

THE SOFTWARE IS PROVIDED "AS IS", WITHOUT WARRANTY OF ANY KIND,
EXPRESS OR IMPLIED, INCLUDING BUT NOT LIMITED TO THE WARRANTIES OF
MERCHANTABILITY, FITNESS FOR A PARTICULAR PURPOSE AND
NONINFRINGEMENT. IN NO EVENT SHALL THE AUTHORS OR COPYRIGHT HOLDERS BE
LIABLE FOR ANY CLAIM, DAMAGES OR OTHER LIABILITY, WHETHER IN AN ACTION
OF CONTRACT, TORT OR OTHERWISE, ARISING FROM, OUT OF OR IN CONNECTION
WITH THE SOFTWARE OR THE USE OR OTHER DEALINGS IN THE SOFTWARE.
\pagebreak

\tableofcontents
\vfill
\pagebreak

\section{Dirac Operator}

The twisted mass Wilson Dirac operator is defined by
\begin{equation}
\left\langle\bar\psi|D_{TW}|\psi\right\rangle =
  \sum_{x,x'}\bar\psi(x)D_{TW}(x,x')\psi(x'),
\end{equation}
where
\begin{eqnarray}
%D_{TW}(x,x')  &= & (m_q + i \mu \gamma_5)\delta_{x,x'}  -  D_W(x,x')
%\label{D_TW}\\
%D_W(x,x') &=& \frac{1}{2}\sum_{\mu=0}^3\left[(1-\gamma_\mu) U_\mu(x)\delta_{x,x'-\hat\mu}
%         +(1+\gamma_\mu) U_\mu^\dagger(x-\hat\mu)\delta_{x,x'+\hat\mu}\right]
%\label{action-W}
D_{TW}(x,x')  &= & (m_q + i \mu \gamma_5)\delta_{x,x'} 
   - \frac{1}{2}\sum_{\mu=0}^3\left[(1-\gamma_\mu) U_\mu(x)\delta_{x,x'-\hat\mu}
         +(1+\gamma_\mu) U_\mu^\dagger(x-\hat\mu)\delta_{x,x'+\hat\mu}\right]
\label{action-W}
\end{eqnarray}
The library assumes that both $\kappa$ and $\mu$ are complex.
\vfill
\pagebreak
\section{Gamma Matrices}
We use the same $\gamma$-matrix basis as Chroma to simplify conversion between
two codes. The Twisted code expects $\gamma_5$ and $\sigma_{\mu\nu}$
to have the following values.

\begin{eqnarray*}
\gamma_0 &=& -\sigma_2 \otimes \sigma_1  =
           \left(\begin{array}{cc}
                0&i\sigma_1\\
                -i\sigma_1&0
             \end{array}\right) = 
           \left(\begin{array}{cccc}
                                 0&0&0&i\\
                                 0&0&i&0\\
                                 0&-i&0&0\\
                                 -i&0&0&0
                           \end{array}\right),\\
\gamma_1 &=& \sigma_2 \otimes \sigma_2  = 
           \left(\begin{array}{cc}
                  0 & -i\sigma_2\\
                  i\sigma_2 & 0
                 \end{array}\right) = 
           \left(\begin{array}{cccc}
                                 0&0&0&-1\\
                                 0&0&1&0\\
                                 0&1&0&0\\
                                 -1&0&0&0
                           \end{array}\right),\\
\gamma_2 &=& -\sigma_2 \otimes \sigma_3  = 
           \left(\begin{array}{cc}
                   0 & i\sigma_3\\
                 -i\sigma_3&0
                 \end{array}\right) = 
           \left(\begin{array}{cccc}
                                 0&0&i&0\\
                                 0&0&0&-i\\
                                 -i&0&0&0\\
                                 0&i&0&0
                           \end{array}\right),\\
\gamma_3 &=& \sigma_1 \otimes 1 =
           \left(\begin{array}{cc}
                  0 & 1\\
                  1 & 0
                 \end{array}\right) = 
           \left(\begin{array}{cccc}
                                 0&0&1&0\\
                                 0&0&0&1\\
                                 1&0&0&0\\
                                 0&1&0&0
                           \end{array}\right),\\
\gamma_5 &=& \gamma_0 \gamma_1 \gamma_2 \gamma_3 = \sigma_3 \otimes 1 =
           \left(\begin{array}{cc}
             1 & 0 \\
             0 & -1
           \end{array}\right) =
           \left(\begin{array}{cccc}
             1 & 0 & 0 & 0\\
             0 & 1 & 0 & 0\\
             0 & 0 &-1 & 0\\
             0 & 0 & 0 &-1
           \end{array}\right).
\end{eqnarray*}
\vfill
\pagebreak
\section{Preconditioning}\label{precond}
We use four dimensional preconditioner to improve convergence of the CG.
Let us color the lattice sites according to the
parity of $x_0+x_1+x_2+x_3$. Then we can rewrite $D_{TW}$ from Eq.~(\ref{D_TW})
as follows:
\begin{equation}\label{EO-form}
D_{TW}=\left(\begin{array}{cc}
A_{oo} & B_{oe}\\
B_{eo} & A_{ee}
\end{array}\right),
\end{equation}
where
\begin{eqnarray}
A_{oo}(x,x') & = & (m_q + i \mu \gamma_5)\delta_{x,x'}, \\
B_{oe}(x,x') & = & -\frac{1}{2}
            \sum_{\mu=0}^{3}\left[(1-\gamma_\mu)U_\mu(x)\delta_{x,x'-\hat\mu}
         +(1+\gamma_\mu)U_\mu^{\dagger}(x-\hat\mu)\delta_{x,x'+\hat\mu}\right],
\end{eqnarray}
and similary for other parity components.
(Hereafter spinor and color indices are suppressed but presumed.)

Let us rewrite Eq.~(\ref{EO-form}) as follows:
\begin{equation}
D_{TW}=
\left(\begin{array}{cc}
I_{oo} & 0\\
B_{eo}A_{oo}^{-1} & I_{ee}
\end{array}\right)
\left(\begin{array}{cc}
A_{oo} & B_{oe}\\
0 & A_{ee}-B_{eo}A_{oo}^{-1}B_{oe}
\end{array}\right).
\end{equation}

To solve the equation
\[
D_{TW}\psi=
\left(\begin{array}{cc}
A_{oo} & B_{oe}\\
B_{eo} & A_{ee}
\end{array}\right)
\left(\begin{array}{c}\psi_e\\\psi_o\end{array}\right) =
   \left(\begin{array}{c}\eta_e\\\eta_o\end{array}\right),
\]
one performs the following steps:
\begin{enumerate}
\item Use $M=A_{ee}-B_{eo}A_{oo}^{-1}B_{oe}$ in the
following.
\item Use $M^\dagger=A_{ee}^\dagger-\left(B_{oe}\right)^\dagger
\left(A_{oo}^{-1}\right)^\dagger
\left(B_{eo}\right)^\dagger$ in the following.
\item Compute
\[
\chi_e =M^{\dagger} \left(\eta_e
            - B_{eo}A_{oo}^{-1}\eta_o\right).
\]
\item Solve
\begin{equation}
M^{\dagger}M\psi_e = \chi_e
\label{normal-eq}
\end{equation}
for $\psi_e$.
\item Compute
\begin{eqnarray}
\psi_o &=& A_{oo}^{-1}\left(\eta_o-B_{oe}\psi_e\right).
\end{eqnarray}
\end{enumerate}

Note, that computing the inverse of $A_{oo}$ is local and can be precomputed:
\begin{eqnarray*}
A_{oo}     &=& \left(\begin{array}{cc}X & 0 \\ 0 & Y\end{array}\right),\\
A_{oo}^{-1} &=& \left(\begin{array}{cc}1/X & 0 \\ 0 & 1/Y\end{array}\right),\\
X &=& m_q + i \mu, \\
Y &=& m_q - i \mu. \\
\end{eqnarray*}
Here $X$ acts on two upper components of the fermion and $Y$ acts on
two lower components.

\end{document}
